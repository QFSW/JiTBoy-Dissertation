\subsection{Project Specification}

The project will consist of developing the following emulators:

\begin{itemize}
    \item \textbf{Interpreter}
    
    The interpreter will emulate the source instructions by inspecting each instruction as required and directly executing the equivalent behaviour. Only a single thread will be used and no translation to the host architecture will be performed. This will largely serve as a performance baseline.
    
    \item \textbf{JIT Runtime}
    
    The JIT runtime will translate each source block from the foreign ISA to the host ISA which will then be executed to emulate the foreign program. Only a single thread will be used and no interpretation will be performed.
    
    \item \textbf{Hybrid Runtime}
    
    The hybrid runtime will utilise both the interpreter and JIT components in tandem with the aim to maximize the performance. At its discretion, it will utilise additional worker threads to JIT translate foreign source blocks to native host machine code blocks. The main thread will execute the translated blocks and will fall back to the interpreter if they are unavailable.
\end{itemize}

The following criterion have been defined for the emulators:

\begin{itemize}
    \item \textbf{Source Architecture}: MIPS-1
    
    MIPS-1 has been selected as the ISA for the source architecture as it is a simple ISA. The MIPS-1 CPU will be simulated to the register and memory level.

    No operating system features will initially be emulated and the project will serve to emulate MIPS-1 programs running on bare metal. Methods to explore abstracting the OS layer in the emulator is left for future work.

    None of the co-processors will be supported as they increase the complexity of the implementation without providing increased insight of value. This means no floating point operations will be supported.
    
    \item \textbf{Host Architecture}: x86
    
    x86 has been selected as the host ISA as it is the most frequently used ISA on modern desktop computers. This allows the project to demonstrate its potential use in a commercial product designed for x86 systems. Furthermore, x86 is a complex instruction set computer (CISC) architecture in contrast to MIPS's reduced instruction set computer (RISC) architecture. This difference allows me to explore more demanding emulation as the more abstraction will be required due to the different natures of the ISAs.
    
    x64 is not considered or supported at this time as it provides extra implementation complexity without providing evaluation results of extra interest.
    
    \item \textbf{Host System}: Windows 10
    
    Due to project details such as calling conventions and executable memory allocation, the project is not cross-platform out of the box without further work. For this reason only a single host system will be supported for the initial implementation to keep the scope manageable. Windows 10 has been selected due to it being the majority OS on desktops in consumer use \cite{desktop-os-share, win-os-share}.
    
    \item \textbf{Emulator Input}: MIPS-1 binaries, MIPS-1 assembly
    
    MIPS-1 binaries will be supported as this allows for compatibility with built binaries from various sources and tools. Direct support for assembly makes development and testing much easier as the emulator can then directly consume the source code. Pseudoinstructions such as \texttt{mov} will be supported for convenience.

    \item \textbf{Microarchitecture Simulation}: None
    
    Microarchitecture simulation can be of great use in research for simulating different CPU designs and investigating their performance and power characteristics. Having said this, it does not aid in the emulation of a foreign ISA and will introduce a large and unnecessary overhead and complexity. Given this, no microarchitecture simulation will be considered for this project.
\end{itemize}