\subsection{Performance}
\label{section:results-perf}

This section will explore the performance characteristics shown by all of the SUTs and will answer the research questions proposed in \autoref{section:req-cap-performance}. Furthermore, the optimisations introduced in \autoref{section:optimisations} will be explored to determine which configurations are optimal and when.

This investigation will be conducted in two steps. First, the generalised performance across all tests will be explored for each emulator; this can be done by observing the relationship between different test statistics (such as performance vs hotness). From this the ideal configuration for each emulator will be determined. This will be covered in \Cref{section:perf-interpreter,section:perf-jit,section:perf-hybrid}.

Next, in \Cref{section:perf-iteration,section:perf-recursion,section:perf-memory}, the performance will be investigated more thoroughly on the bespoke set of test suites. Unlike the functionality tests, these test suites are specifically designed to stress the SUTs in different ways and have been designed to answer the various performance research questions. The test suites of interest are as follows:

\begin{itemize}
    \item \textbf{\texttt{fibonacci(n)}}
    
    Recursive program that computes \texttt{fibonacci(n)} designed to explore the performance of highly recursive program. Uses a mixture of ALU and memory operations. The computational complexity is $\mathcal{O}(2^n)$ and thus we can expect to see very long and intensive tests for larger $n$.
    
    \item \textbf{\texttt{fibonacci\_rep[k](n)}}
    
    The same as \texttt{fibonacci(n)}, however $k$ unique instantions of the \texttt{fibonacci} function are generated. This helps investigate how performance scales with more unique code. The values of $k$ used were $10$ and $100$.
    
    \item \textbf{\texttt{factorial(n)}}
    
    Recursive program that computes \texttt{factorial(n)}. Since the complexity is $\mathcal{O}(n)$ and $n!$ grows exponentially with respect to $n$, this test suite will not lead to long intensive tests.

    \item \textbf{\texttt{memcpy(n)}}
    
    Iterative program that copies $n$ bytes from one location in memory to another. The very high proportion of memory instructions touching a large space of memory will stress the memory map.
    
    \item \textbf{\texttt{memcpyw(n)}}
    
    The same as \texttt{memcpy(n)} however words will be touched and copied instead of individual bytes.
    
    \item \textbf{\texttt{primal(n)}}
    
    An iterative program that determines if $n$ is primal. With no memory instructions present this should result in a very hot ALU heavy loop. Since the complexity is $\mathcal{O}(n)$, the values of $n$ supplied to the test suite will grow on the order of $2^n$; all values of $n$ used will be primal.
    
    \item \textbf{\texttt{unroll(n/m)}}
    
    A program loop of $n$ iterations that has been unrolled into $n/m$ chunks such that the unrolled program has $m$ iterations. Designed to investigate how the trade-off of unrolling varies with each SUT.
\end{itemize}

\YIComment{Talk about fib\_rep tests}

\subfile{interpreter}
\subfile{jit}
\subfile{hybrid}
\subfile{iteration}
\subfile{recursion}
\subfile{memory}
