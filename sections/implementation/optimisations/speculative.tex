\subsubsection{Speculative Compilation (\texttt{-S})}

The hybrid architecture described in \autoref{section:hybrid-arch} allows the system to use extra worker threads to asynchronously JIT compile source blocks, deferring the work away from the main thread; whilst this is helpful, the just in time compilation could in some cases be seen as just too late. Blocks are scheduled for JIT compilation when they are needed, meaning they are not ready at that time and the interpreter fallback must be used.

Speculative compilation, denoted by \texttt{-S}, can help remedy this by building off of the work provided by direct linking. Whereas \texttt{-L} aims to relink all unresolved jumps present in the compiled blocks, speculative compilation instead attempts to schedule the destination blocks for JIT compilation whilst they remain unresolved; this allows the system to speculatively compile blocks that \textit{might} be executed from the blocks currently in the translation table. The current speculation algorithm attempts to schedule the destinations of all unresolved jumps for compilation; a more complex speculation criteria could yield better results.

This would allow the system to translate the blocks ahead of time, potentially before they are needed, reducing the amount of execution that must be delegated to the interpreter fallback. On the other hand, this could fill the compilation queue with low priority or completely wasteful blocks; increased contention could increase the time before important blocks are compiled, degrading the overall performance.

Speculative compilation is compatible with direct linking, and is only available for the hybrid emulator. It would serve no benefit on the standard JIT emulator as without asynchronous compilation there is nothing to gain by compiling blocks early.