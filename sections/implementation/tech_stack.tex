\subsection{Technology Stack}

The technology stack used to develop the project was the first major decision required before further work could take place; the component of the highest importance was the primary programming language. The following list details a set of requirements that the language must meet in order to be a good fit for the project:

\begin{itemize}
    \item \textbf{Windows 10 and x86 Support}
    
    Cross-platform support and portability is highly desirable for the language of choice; given the project scope is restricted to Windows 10 and x86 these are the only strict requirements.

    \item \textbf{Strong Typing}
    
    Due to the size, complexity and length of this project, I have included strong typing as a requirement. Strong typing greatly increases the readability by self documenting the codebase and the reliability improvements due to compile time type errors are paramount.

    \item \textbf{No Garbage Collection}
    
    Garbage collection (GC) is an automatic memory management scheme that does not require the user to manually free resources. While this is very flexible and easy to use, the unreliable performance implications of a GC are unacceptable for this project.

    \item \textbf{Unmanaged Function Pointers}
    
    The ability to create an unmanaged pointer to a function allows us to obtain the address of the native machine code for the function. This allows the JIT to compile direct function calls to runtime functions written in the programming language of choice.

    \item \textbf{Executable Memory Allocation}
    
    Executable memory cannot be allocated through the typical means that languages and programmers use for general memory allocation such as \texttt{malloc}. Special OS APIs are required and thus the language must be able to use these APIs. A native call will give the highest performance however an interop call with marshalling would still be viable.

    \item \textbf{Reinterpret Cast}
    
    The compiled block of x86 code will need to be reinterpreted as a function pointer and executed as a function in order to execute the output of the JIT compiler.
\end{itemize}

\autoref{tbl:lang-reqs} contains a matrix of many programming languages and which requirements they meet, including both languages that I have previous project experience in and popular \cite{stackoverflow-survey} but unfamiliar languages.

\newcommand{\y}{\color{success-green}{\cmark}}
\newcommand{\n}{\color{fail-red}{\xmark}}

\begin{table}[h] 
    \centering
    \begin{tabular}{r|cccccc}
        & \rot{45}{Windows 10 and x86 Support} & \rot{45}{Strong Typing} & \rot{45}{No Garbage Collection} & \rot{45}{Unmanaged Function Pointers} & \rot{45}{Executable Memory Allocation} & \rot{45}{Reinterpret Cast} \\
        \midrule
        x86 Assembly & \y & \n & \y & \y & \y & \y \\
        C            & \y & \y & \y & \y & \y & \y \\
        C++          & \y & \y & \y & \y & \y & \y \\
        C\#          & \y & \y & \n & \y & \y & \y \\
        Java         & \y & \y & \n & \n & \n & \n \\
        JavaScript   & \y & \n & \n & \n & \n & \n \\
        TypeScript   & \y & \y & \n & \n & \n & \n \\
        Python       & \y & \n & \n & \n & \n & \n \\
        Rust         & \y & \y & \y & \y & \y & \y \\
        Haskell      & \y & \y & \n & \y & \y & \y \\
        F\#          & \y & \y & \n & \y & \y & \y \\
        Go           & \y & \y & \n & \n & \y & \y \\
        Swift        & \n & \y & \n & \y & \y & \y \\
        Objective-C  & \n & \y & \y & \y & \y & \y \\
        \bottomrule
    \end{tabular}
    \caption{Various potential candidates for the language of implementation and the requirements they meet.}
    \label{tbl:lang-reqs}
\end{table}

The only languages that met all of the requirements were Rust, C, and C++. Rust provides a very safe yet fast memory model via its ownership model \cite{rust-ownership} and has also recently become a community favourite \cite{stackoverflow-survey}. This said, it is a very new language with little industrial presence \cite{stackoverflow-survey}.  C provides little in the way of encapsulation making it difficult to write predictable and mantainable code.

This leaves us with C++, a language that provides very good performance with relatively high safety if used correctly. While the underlying memory model is just as unsafe as C's, many modern constructs such as smart pointers \cite{cpp-smart-ptrs} and RAII \cite{cpp-raii} help avoid typical issues like memory leaks while maintaning high performance. Due to these reasons, and my current experience with the language, I believed it was the best fit for this project.

As the project is scoped for Windows 10, MSVC \& Visual Studio 2019 was used as the C++ compiler toolchain.

Functional languages like Haskell and F\# were considered due to their natural affinity for developing components such as the parser or intermediate representation (IR) based optimisers; while they would have been a good fit for such components, they would have been less viable for core components such as the compiler due to reasons discussed previously. A multi-process solution was not desirable so they were not considered further.

Python was utilised for the analysis and visualisation of the test results. Python is a very powerful scripting language with strong community support making it an ideal choice for smaller scripts that do not require high performance. The package matplotlib was used for visualisation due to its power and ease of use.

MIPS assembly was used for extensive testing as discussed in detail by \autoref{section:testing}.

The usage of each language is detailed in \autoref{tbl:lang-lines}.

\begin{table}[h] 
    \centering
    \begin{tabular}{l|rrrr}
        \toprule
        Language & Usage & Files & Lines of Code \\
        \midrule
        C++ & Core Project & 108 & 8031 \\
        Python & Analysis & 14 & 791 \\
        DOS Batch & Automation & 1 & 4 \\
        MIPS Assembly & Testing & 567 & 123319 \\
        \bottomrule
    \end{tabular}
    \caption{Languages used in the project and the number of lines of code and file count for each language. Lines of code are inclusive of comments and blank lines. 3rd party libraries are not included.}
    \label{tbl:lang-lines}
\end{table}