\begin{abstract}
    Emulation of a binary program written for a foreign instruction set architecture (ISA) is of great importance for academic, industrial and recreational usage. Among many uses, emulation allows the continued usage of legacy software on new ISAs. Traditional emulation using interpretation does not always allow for emulation at an acceptable speed, especially for complex or real-time programs.
    
    This project explores the utilisation of just in time (JIT) compilation to accelerate the emulation through dynamic binary translation (DBT). The project also explores hybridising the interpreter and JIT techniques to create a concurrent hybrid emulator that exploits the multiple logical processors available on all modern consumer CPUs to accelerate performance. The emulators were developed to emulate foreign programs written for the MIPS-1 ISA on a modern x86 CPU.

    The JIT emulator developed was able to emulate the provided benchmarks at speeds far exceeding the interpreter. In some cases the JIT emulator was able to emulate programs at over 2000 mega instructions per second (mips) while the interpreter peaked at less than 100 mips. The hybrid was able to simultaneously beat the interpreter for heavy workloads and the JIT for light workloads.
\end{abstract}