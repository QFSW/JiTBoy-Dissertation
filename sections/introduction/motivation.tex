\subsection{Motivation}

Emulation of software written for a different instruction set architecture (ISA) is frequently required throughout the industry both by developers and non-technical end users. This ability to emulate software both correctly and efficiently is greatly prevalent in recent times.

Emulation allows developers and researchers to develop, debug and utilise software for a foreign ISA without the need for additional hardware, which can significantly reduce costs, but only if the emulation is adequately fast to substitute for the real hardware. Popular examples of such include QEMU \cite{qemu}.

Another use of emulation is the ability to emulate legacy software that cannot be feasibly rewritten. Many industries use old software as it is too expensive or difficult to rebuild, and in some cases, such as those using IBM mainframes \cite{ibm-mainframe}, can end up locked into restrictive and/or old hardware choices due to their programs being written for ISAs that are no longer popular.

Recently, this has become important as ever for many average consumers. With recent industry events, notably including Apple moving their desktop line to use their in house ARM based Apple Silicon CPUs \cite{apple-silicon,rosetta2}, many consumers are now finding themselves needing the ability to emulate old, foreign software on their new hardware. It is in the industry's utmost interest that this emulation is fast and accurate, otherwise the user experience of their new hardware offerings will be severely affected. Apple developed Rosetta 2 \cite{rosetta2} to solve this problem.

Another large motivation for efficient emulation of foreign code is in the enthusiast space; many enthusiasts want to run deprecated software such as old video games written for hardware that is no longer manufactured, and thus must resort to emulation. Popular examples of such emulators include Dolphin \cite{dolphin}, an emulator for the Nintendo GameCube and Wii systems, and PCSX2 \cite{PCSX2}, an emulator for the Sony PlayStation 2 system. In this case performance is key due to the real-time nature of video games.

Emulation of old games has also been used in the industry by 1st parties as a way to sell older games without rebuilding them. Examples of this include Nintendo's release of \emph{Super Mario 3D All Stars} \cite{mario-emulation} and the Microsoft Xbox One's ability to play select Xbox 360 games \cite{xbox360-emulation}.

\JW{This first subsection has too many short paragraphs. A bit of restructuring would help to make your argument clearer. If part of this subsection is `here is a list of use-cases for emulation', you might consider using a bulleted list, for instance.}
