\subsection{Motivation}

Emulation of software written for a different instruction set architecture (ISA) is frequently required throughout the industry both by developers and non-technical end users. This ability to emulate software both correctly and efficiently has only grown more prevalent in recent times.

Emulation allows developers and researchers to develop, debug and utilise software for a foreign ISA without the need for additional hardware, which can significantly reduce costs, but only if the emulation is adequately fast to substitute for the real hardware.

More widespread, however, is the requirement to emulate legacy software, that cannot be feasibly rewritten. Many industries use old software as it is too expensive or difficult to rebuild, and in some cases such as those using IBM datacenters, end up locked into restrictive and old hardware choices due to their programs being written for no longer used ISAs.

Recently, this has become important as ever for many average consumers. With recent industry events, notably including Apple moving their desktop line to use ARM based CPUs, many consumers are now finding themselves needing the ability to emulate old, foreign software on their new hardware. It is in the industry's upmost interest that this emulation is fast and accurate, otherwise the user experience of their new hardware offerings will be severely affected.

Another large motivation for efficient emulation of foreign code is in the enthusiast space; many enthusiasts want to run deprecated software such as old video games written for hardware that is no longer manufactured, and thus must resort to emulation. In this case performance is key due to the real-time nature of video games.
