\subsection{Motivation}

Emulation of software written for a different instruction set architecture (ISA) is frequently required throughout the industry both by developers and non-technical end users. This ability to emulate software both correctly and efficiently has only grown more prevalent in recent times. \JWComment{Can you give a citation to justify your claim that emulation has grown more prevalent recently?} Popular emulators include QEMU \cite{qemu} and Rosetta 2 \cite{rosetta2}.

Emulation allows developers and researchers to develop, debug and utilise software for a foreign ISA without the need for additional hardware, which can significantly reduce costs, but only if the emulation is adequately fast to substitute for the real hardware.

More widespread, however, is the requirement to emulate legacy software, that cannot be feasibly rewritten.\JWComment{Can you give a citation to justify that legacy emulation is more widespread than future emulation? If not, maybe just say that it's another use of emulation.} Many industries use old software as it is too expensive or difficult to rebuild, and in some cases, such as those using IBM mainframes \cite{ibm-mainframe}, can end up locked into restrictive and/or old hardware choices due to their programs being written for ISAs that are no longer popular.

Recently, this has become important as ever for many average consumers. With recent industry events, notably including Apple moving their desktop line to use their in house ARM based Apple Silicon CPUs \cite{apple-silicon,rosetta2}, many consumers are now finding themselves needing the ability to emulate old, foreign software on their new hardware. It is in the industry's utmost interest that this emulation is fast and accurate, otherwise the user experience of their new hardware offerings will be severely affected.

Another large motivation for efficient emulation of foreign code is in the enthusiast space; many enthusiasts want to run deprecated software such as old video games written for hardware that is no longer manufactured, and thus must resort to emulation. Popular examples of such emulators include Dolphin \cite{dolphin}, an emulator for the Nintendo GameCube and Wii systems, and PCSX2 \cite{PCSX2}, an emulator for the Sony PlayStation 2 system. In this case performance is key due to the real-time nature of video games.

Emulation of old games has also been used in the industry by 1st parties as a way to sell older games without rebuilding them. Examples of this include Nintendo's release of \emph{Super Mario 3D All Stars} \cite{mario-emulation} and the Microsoft Xbox One's ability to play select Xbox 360 games \cite{xbox360-emulation}.
