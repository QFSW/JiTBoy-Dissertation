\subsection{Hybrid Emulation}
\label{section:intro-hybrid}

While the JIT emulation should perform better than the interpreter in most cases due to its far reduced per-instruction overhead, as explored in \autoref{section:intro-jit}, this won't always be the case. Since JIT compilation involves a far larger overhead due to the translation required, we would expect the JIT emulator to have worse responsiveness than the interpreter and perform poorer for light workloads.

In an attempt to mitigate the weakness of either technique we propose the JIT-Interpreter hybrid emulator. Conceptually, the hybrid would asynchronously JIT compile source blocks off of the main thread whilst executing code on the main thread. In the case that a block isn't compiled yet, it would use the interpreter as a fallback.

In theory, this would avoid the large overhead associating with JIT compiling a source block; the compilation would still occurr, but by deferring it from the main thread the emulator can simultaneously use the interpreter for its low overhead. At the same time, the emulator should be able to reach the same peak performance as the JIT emulator as it will eventually compile all blocks and no longer use the interpreter fallback.

Furthermore, neither the interpreter or the JIT emulator can trivially use multiple CPU cores. This leaves much to be desired as all modern desktop systems contain multiple logical processors. The hybrid emulator can potentially make better utilisation of these logical processors by concurrently JIT compiling multiple source blocks at once on several threads. In theory, this should allow for higher performance than the JIT emulator as it is able to perform more work at once.