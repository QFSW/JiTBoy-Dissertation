\subsection{Report Structure}

The report first explores the current background and literature on binary emulation in \autoref{section:background}.

Next, the project is scoped. Specific details about the functionality and extent of the project are determined and the research questions used to describe the success of the project are proposed. This is covered in \autoref{section:req-caps}.

The implementation of the project is then covered in \autoref{section:implementation}. This begins with the technology stack used and the high level architectural design of the key components such as the various emulators. It then dives into the more interesting implementation details.

With implementation complete, \autoref{section:testing} focuses on testing. This section describes the methodology behind the testing and the framework design. It describes the different metrics and criteria developed, but will not focus on any individual tests or results. This is covered by \autoref{section:results} which shows and evaluates the test results and aims to answer the research questions proposed in \autoref{section:req-caps}.

Finally, \autoref{section:conclusions} concludes the report and summarises the evaluation performed in \autoref{section:results}. This section covers the implications of the results before exploring further work suitable for the project.

Additionally, \autoref{section:user-guide} contains a brief user guide for building and running the finished project. \autoref{section:legal} clarifies any potential issues regarding the legality of the project.