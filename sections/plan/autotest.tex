\subsection{Automated Test Generation}

In order to produce a more statistically robust analysis and evaluation, much more performance test cases will be required, specifically those with different sizes, structures and complexities. Whilst some tests will be written by hand to explore specific details, writing hundreds of thousands of tests by hand can quickly become infeasible, even with generators.

Csmith \cite{csmith} is a tool that can generate an unlimited number of valid C programs. This combined with GCC \cite{gcc} cross compiled for MIPS \cite{mips-gcc} can be used to generate an unlimited number of valid MIPS assembly/binary files. These tests can then be added to the test directory and consumed by the existing testing framework.

Currently, the test framework is not directly compatible with the output of GCC for a few reasons:

\begin{itemize}
    \item \textbf{Compiler Directives}
    
    The parser will be upgraded to consume the compiler directives either ignoring them or actioning them if applicable.

    \item \textbf{Pseudoinstructions}
    
    GCC makes use of MIPS pseudoinstructions such as \texttt{mov}. The parser will be extended to support these.

    \item \textbf{No Entry Point}
    
    The generated assembly does not have a compiled entry point and instead the \texttt{.ent} directive is used to indicate the main function as being an entry point. A simple post processor script can be used to auto generate entry point assembly at the beginning of the file that calls the main function appropriately.
\end{itemize}

The output of Csmith is not guaranteed to terminate \cite{csmith-paper}. This would cause the test runner to stall indefinitely whilst trying to measure the performance. To remedy this, a post-processing step can be added to the test generation process. Once a test has been completely generated by the previous steps, it can be invoked directly on the emulator with a timeout; if the timeout is exceeded then the test will be discarded from the suite and re-generated.

This will be complete by the end of April.