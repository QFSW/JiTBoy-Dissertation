\subsection{Direct Block Linking}

Currently no \texttt{x86} blocks produced by the JIT compiler contain jump instructions. This is because the destination of the jump cannot be known at compile time as it may be pointing to a not yet translated MIPS source block. Due to this, all blocks instead terminate with a return code indicating the desired target MIPS address.

In the steady state execution, where all blocks are translated, this results in a dispatch loop where blocks must first return to the runtime, lookup the corresponding target block and execute it instead of directly jumping. This dispatch loop can cause a significant performance impact, with profiling showing up to 40\% of all execution time spent in the dispatch loop on some test suites.

To mitigate this, a retroactive block patching system will be employed. This will rewrite already compiled \texttt{x86} blocks to jump directly the destination block once the aforementioned destination block has been generate. This will further improve the performance of the JIT emulator in long and hot programs.

This will be complete by the end of May.