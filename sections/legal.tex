\section{Ethical, Legal and Safety Plan}

At times emulation finds itself in a gray area legally, usually due to 2 reasons:

\begin{itemize}
    \item \textbf{Unlawful redistribution of protected Intellectual Property (IP)} %
    
    Some emulators require protected pieces of IP in order to function correctly. One example of this is PCSX2 requiring a copy of the PlayStation 2 BIOS in order to operate \cite{PCSX2-getting-started}; distribution of this BIOS would be illegal and thus PCSX2 does not come with the BIOS included. Emulators for recreational purposes are also believed to be illegal by some as they allow for piracy of protected IP. This is a misconception as it is the unlawful distribution of the protected IP that is illegal and not the ability to emulate them. Neither of these are a concern in my case as both the emulator and the test programs are written myself and/or sourced under sufficiently permissive licenses.

    \item \textbf{Utilising protected IP to build the emulator} %
    
    In some cases protected IP, such as code, design or otherwise protected documents, are utilised when building an emulator. This poses an issue at is unlicensed usage of the protected IP. In my case all documentation and resources used to build the emulator are legally publicly available and thus this is not a concern.
\end{itemize}

Given this discussion, there are no identifiable ethical, legal or safety risks associated with the project.