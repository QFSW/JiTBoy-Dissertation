\subsection{Industry}

Apple developed Rosetta \cite{apple-rosetta} to aid in the transition from their earlier PowerPC based Macs to the, at the time, new Intel based macs. This software was thus a binary translator from PowerPC to x86, based off of QuickTransit by Transitive Corporation \cite{cnet-rosetta}. Rosetta was unable to emulate programs that utilised AltiVec instructions (a SIMD ISA extension). Rosetta was shown to have poor performance when emulating computationally heavy programs \cite{rosetta-perf}.

Microsoft developed an x86 to ARM emulator as part of Windows 10 for ARM \cite{docs-win10-arm-emu}, utilising the WOW64 \cite{WOW64} layer of Windows 10. Recently, this emulation layer was extended to support x64 binaries running under Windows 10 for ARM \cite{win10-arm-x64-emu}. The emulator utilises a JIT compiler to convert blocks of x86 to native ARM code, which are cached by a background service \cite{docs-win10-arm-emu, blackberry-win10-arm-emu}. This artefact caching allows subsequent runs to benefit from the compilation incurred in initial runs, helping to amortise the initial compilation cost and improving the performance of subsequent executions. Despite this, the emulation was shown to have very poor performance in comparison to native applications \cite{win10-arm-x64-emu-perf1, win10-arm-x64-emu-perf2}.

Recently, Apple developed Rosetta 2 \cite{rosetta2} in house to ait in the transition from the Intel based Macs to their new ARM based Apple Silicon Macs \cite{rosetta2, apple-silicon}. Rosetta 2 reportedly employs an AOT translation technique, in which portions of the program are converted before it is first executed \cite{rosetta2-aot, ars-technica-big-sur}. This technique involves statically analysing portions of code in the binary and translating them ahead of time which are stored in an artefact cache; any new code encountered at runtime falls back to a JIT. This is a similar technique to that employed by Microsoft, but with the addition of the static AOT lookahead. Whilst this technique causes longer than desirable initial program boot times \cite{rosetta2-slow-launch}, Rosetta 2 has been shown to have much higher emulation performance than either the original Rosetta or Microsofts ARM emulation \cite{rosetta2-perf}. The Apple M1 SOC reportedly includes hardware support for the x86 memory consistency model, which may attribute to Rosetta 2's performanace success \cite{rosetta2-infoq}. Just like the original Rosetta, Rosetta 2 is unable to emulate programs utilising new SIMD ISA extensions such as AVX, AVX2, and AVX512 \cite{rosetta2}.